\documentclass[12pt,a4paper]{article}

\usepackage[margin=1in]{geometry}
\usepackage{graphicx}
\usepackage{caption}
\usepackage{subcaption}
\usepackage{amsmath}
\usepackage{hyperref}
\usepackage[english]{babel}
\usepackage{booktabs}

\title{Design and Analysis of an Intelligent Traffic Monitoring System using Computer Vision}
\author{
Adarsh Vashist\\
Department of Computer Science \& Engineering\\
Dayananda Sagar College of Engineering, Bengaluru
}
\date{January 2026}

\begin{document}

\maketitle

\begin{abstract}
This mini-report presents the design and preliminary analysis of an intelligent traffic monitoring system based on computer vision techniques. The system detects vehicles, counts them, estimates density, and identifies congestion using real-time video feeds. YOLOv8 and OpenCV are used for object detection and tracking. Early results show promising accuracy in urban Indian traffic scenarios.
\end{abstract}

\tableofcontents

\section{Introduction}
Traffic congestion is a major challenge in metropolitan cities such as Bengaluru. Traditional traffic monitoring approaches using inductive loops or manual observation are expensive, intrusive, and prone to errors.

Computer vision provides a scalable and non-intrusive alternative. The proposed system performs the following tasks:
\begin{itemize}
    \item Real-time vehicle detection and classification
    \item Multi-object vehicle tracking
    \item Traffic density estimation and congestion detection
\end{itemize}

\section{System Design}

\subsection{Architecture}
The system follows a modular processing pipeline consisting of video acquisition, vehicle detection, tracking, and traffic analysis.

\subsection{Object Detection Module}
The YOLOv8 nano model is used for vehicle detection due to its balance between accuracy and real-time performance. Detected vehicles are tracked across frames using a lightweight multi-object tracking algorithm.

\section{Implementation Details}

\subsection{Hardware and Software}
\begin{itemize}
    \item Camera: 1080p IP camera operating at 30 fps
    \item Processing Unit: NVIDIA Jetson Nano or laptop with mid-range GPU
    \item Libraries: OpenCV, Ultralytics YOLOv8, NumPy, Pandas
\end{itemize}

\subsection{Performance Metrics}
Experiments were conducted on a 10-minute urban traffic video.

\begin{table}[h]
\centering
\caption{Detection and tracking performance}
\begin{tabular}{l c c}
\toprule
Metric & Value & Description \\
\midrule
mAP@0.5 & 0.84 & Detection accuracy \\
MOTA & 0.76 & Tracking accuracy \\
FPS & 28--32 & Real-time processing \\
\bottomrule
\end{tabular}
\end{table}

\section{Conclusion}
This work demonstrates that lightweight computer vision models can be effectively used for intelligent traffic monitoring in smart cities. The system achieves real-time performance with satisfactory accuracy while using low-cost hardware. Future improvements include multi-camera integration and adaptive traffic signal control.

\end{document}
