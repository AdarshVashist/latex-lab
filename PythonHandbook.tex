\documentclass[10pt,a4paper]{article}
\usepackage[margin=0.7in]{geometry}
\usepackage{listings}
\usepackage{xcolor}
\usepackage{titlesec}

\lstset{
    basicstyle=\ttfamily\small,
    keywordstyle=\color{blue}\bfseries,
    commentstyle=\color{green!60!black}\itshape,
    stringstyle=\color{red},
    numbers=left,
    numberstyle=\tiny\color{gray},
    stepnumber=1,
    breaklines=true,
    frame=single,
    tabsize=4,
    language=Python
}

\titleformat{\section}{\large\bfseries}{\thesection}{1em}{}

\begin{document}

\begin{center}
    {\LARGE Python 3 Quick Reference} \\[4pt]
    {\small Adarsh \quad | \quad January 2026}
\end{center}

\section{Basics}
\begin{lstlisting}
# Hello World
print("Hello, World!")

# Variables
x = 10          # int
name = "Adarsh" # str
pi = 3.14       # float
is_student = True

# Type conversion
str(42)    # '42'
int("100") # 100
\end{lstlisting}

\section{Control Flow}
\begin{lstlisting}
if x > 0:
    print("Positive")
elif x < 0:
    print("Negative")
else:
    print("Zero")

for i in range(5):
    print(i)

while x > 0:
    x -= 1
\end{lstlisting}

\section{Lists \& Dictionaries}
\begin{lstlisting}
fruits = ["apple", "banana", "cherry"]
fruits.append("date")
print(fruits[1])          # banana

person = {"name": "Adarsh", "age": 22, "city": "Bengaluru"}
print(person["city"])
person["skills"] = ["Python", "LaTeX"]
\end{lstlisting}

\section{Functions}
\begin{lstlisting}
def greet(name="Guest"):
    return f"Hello, {name}!"

print(greet("Adarsh"))   # Hello, Adarsh!
print(greet())           # Hello, Guest!

def square(n): return n*n
\end{lstlisting}

\end{document}
